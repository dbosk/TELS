\section{Preliminary project idea}

% # Preliminary project idea (1000 words)
%
% ## Contents
%
% - Tentative aims

If admitted to the programme, I intend to work on an idea that I've had for a 
while: \emph{Technology Enhanced Learning Studies}.
It's a combination of {\ac{TEL}} and {\ac{LS}}.
The main idea is to use the reusability and, consequently, the reproducibility 
of interactive, digital teaching tools to perform scalable research-based 
digital teaching development using learning studies~\autocite{LearningStudy} as 
the underlying research methodology.


\subsection{Learning studies}

Learning studies~\autocite{LearningStudy} was developed by 
\citeauthor{LearningStudy} and colleagues.
The core concept originates from the Japanese lesson study, but was 
complemented by Marton's variation theory of learning and extended by \ac{DBR}.

Variation theory~\autocite{VariationTheory} allows us to analyze teaching 
(recorded classroom teaching, textbooks, assignments, etc.) to determine if 
learning is possible given the students' assumed prerequisites.
Variation theory can thus also guide development of new materials.

\Ac{DBR}~\autocite{DesignBasedResearch} allows us to iterate through designs of 
teaching, making incremental improvements.
The idea of learning study is to use variation theory for initial design, 
analysis and redesign of each iteration.
Each teaching step requires a pre- and post-test of students' knowledge to 
measure any learning.


\subsection{Technology-enhanced learning}

\Textcite{MeaningsAreAcquired} tested variation theory by reproducing a failed 
learning situation using a teaching tool.
In a classroom teaching situation, the teacher had accidentally varied 
something that was supposed to be invariant according to the theory.
The post-test showed that the students didn't learn as intended by the teacher, 
but rather didn't learn as expected according to the theory.
They wanted to reproduce this situation.
They did that by implementing the classroom situation as a computer program 
that the students interacted with.
In one group, the program acted in line with theory; in the other, the program 
made the teacher's mistake.
Indeed, the theory held true, they found the same results.

There are several interesting points with this example, but we'll only cover 
some of them.
The first point I'd like to make, is that the teacher improvising can be good 
and bad.
In the example above, it's bad, but there are other examples where it's 
good~\autocite[cf.][for good examples]{NecessaryConditionsOfLearning}.
Interactive teaching tools (or \ac{TEL} tools), such as Online Learning 
Initiative~\autocite{OLIstatistics} and FeedbackFruits\footnote{%
  Both of which we have access to at KTH.
}, can be used to do such programmed interactive teaching as 
\citeauthor{MeaningsAreAcquired} did above.

My second point is that we need to measure learning to see what parts of our 
teaching has effect.
The students must pass the exam, so they will make sure that they do that, with 
or without our help.
They will discuss with each other, find complementary sources, find better 
sources altogether, and use those for studying instead of our teaching.
We can't measure this with any course evaluation, that too blunt a tool.
We can't use the course assessment either, we don't know if the students 
studied with only the course material or if they get taught by a third party 
(YouTube, StackExchange, etc.).

My last point is about \ac{TEL} tools in general.
\Ac{TEL} tools can easily and suitably replace the traditional lecture.
Few students dare to ask questions during lectures, many students are lost and 
confused before we reach the end of the lecture.
Many students have pointed out the advantage of the lectures during the 
pandemic as getting the video recording, so that they can watch it at their own 
pace, pause or rewind as needed.
Tools like FeedbackFruits allows the students to ask questions, discuss content 
during the videos --- asynchronously with the teacher or their peers.
FeedbackFruits can additionally integrate quizzes into a video, so that 
students can test their understanding and the teacher can see how the students 
are doing.


\subsection{Technology-enhanced learning-studies}

% - Intended results

\paragraph{Intention}

The idea of Technology-Enhanced Learning Studies is to systematically work with 
\ac{TEL} tools and learning studies to continuously research the teaching in 
our courses while we give them.
For instance, we can integrate pre- and post-test of an interactive video into 
the video itself, decreasing the risk of the students getting taught by a third 
party or that there's a significant delay between the teaching and the tests.
If we do this with other parts too, we can track where most learning happens 
and find inefficiencies in our teaching.
However, there is some work to do.

% - What knowledge is needed

\paragraph{What to investigate}

For starters, what should the pre-test look like?
For instance, if the students \emph{learn} something from the pre-test, what 
would that mean for the evaluation of the teaching?
Or should the pre-test be considered part of the teaching?
Probably, it would make sense since that type of construction increases 
learning and retention~\autocite{Szekely1950,BransfordSchwartz1999}.
But then, what does the pre-test say about the learning material?
How do we know if not only the pre-test was sufficient for the students' 
learning?
But maybe that doesn't matter?

Another question, less philosophical and more geared towards practice: can we 
develop a framework for how to easily integrate the pre- and post-tests into 
our teaching using the teaching tools available (Canvas, FeedbackFruits, OLI)?

A perhaps more interesting question, how can we systematically record the 
students' mistakes, or rather misconceptions, to learn from them?
The core of variation theory is that when teaching we must introduce variation 
in one aspect of a learning objective at a time.
The theory postulates\footnote{%
  But with significant empirical support!
} that each misconception correspond to not having discerned one or more 
aspects of the learning objective.
These should thus be identified and guide systematic development of the 
teaching material.

% - Plans for the work
% - Collaborators

%\paragraph{Collaborators}
%
%I already do work on OLI together with Ric Glassey (EECS TCS) and Olle Bälter 
%(EECS MID).
%That work would also benefit from this.
%
%I also do work with former colleagues of mine at Mid Sweden University on using 
%FeedbackFruits to improve teaching and I plan to try this method with them.
%
%I would like to expand my collaborations.
%There are two groups that are interesting:
%Ference Marton's and Angelika Kullberg's groups at Gothenburg University and
%UpCERG at Uppsala University.
%Variation theory and learning study originates from Marton's group.
%Kullberg has worked a lot with learning studies in mathematics.
%UpCERG (e.g.~Anna Eckerdal) has done some research using variation theory in 
%the area of computer science, but the results are limited.
%Neither of them has had the strong technological aspect that I propose.


\subsection{How can this project contribute to KTH?}

% ## Motivation
%
% - How can the work contribute to KTH?
% - Why is this important, what needs are addressed?
% - What can my work contribute to the education programmes and the educational 
%   environment?
% - How does this project connect to the desired development at KTH and the 
%   school? According to
%     - current visions,
%     - development plans,
%     - action plans?
%
% - Life-long learning
% - Student preferences

Many things suggest that KTH must transition its teaching post-covid.
Partly visions, e.g.~more digital teaching.
But also the students like to have video complement, or even replace, the 
traditional lectures.
What I find the most interesting though, is the life-long learning perspective.
There are would-be students who simply cannot study if the only option is the 
traditional campus studies\footnote{%
  I've had many such students while teaching distance courses given to distance 
  study programmes.
  I've had one such student at KTH, he started because he knew that with the 
  pandemic, KTH would be forced to do distance education.
}.

This project will provide a framework for working with education technologies 
that digitalize teaching, that we have access too and use them in the best 
possible way from both a teaching and research perspective.

%Why would we need to always to these pre- and post-tests, even when teaching is 
%working?
%Because prerequisites changes:
%When the teacher changes in a prerequisite course, s/he might no longer cover 
%that exact example that the students needed to see to understand our course.
%When we ourselves updated the preceding module in the same course, we might 
%have changed a something so that a critical aspect is no longer provided to the 
%students.
%When we gave that lecture, we didn't have time to cover the last example in the 
%slides, thus preventing the students from 
%learning~\autocite[cf.][above]{MeaningsAreAcquired}.
