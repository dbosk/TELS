\section{Self-evaluation}

% 1. Demonstrates skill, experience and creativity in a range of teaching 
%    activities.

\paragraph{Background}

I studied the CL programme at KTH and SU between 2006 and 2011, focus on 
mathematics and computer science for upper secondary and adult education.
So I hold a degree Master of Education as well as a Master of Science in 
Computer Science\footnote{%
  And soon a PhD in Computer Science too.
}
After graduation I started working as lecturer of computer science at Mid 
Sweden University, from 2011 until 2020.
In 2014 I started my PhD in computer science at KTH, focus on security and 
privacy.
I kept my lecturer position in parallel until 2020, when I started as lecturer 
full-time at KTH EECS.

I have taught extensively throughout the years.
I have taught both on campus and online, mostly online in fact, entirely online 
from 2015 onwards.
The students have ranged from computer science majors, to the more 
socio-technical industrial engineering and management students all the way to 
the most maths-aversive students.

\subsection{Experience of different students}

I find the students on the distance programmes the more interesting to teach.
In many cases, those students had worked for many years before deciding to 
change track and started studying.
This is in stark contrast to the campus programmes where only fresh graduates 
from the upper secondary schools can study.

Some students were studying a master's programme on distance.
The last time I gave my information security course for those students, one of 
them was working (in parallel to studies) in a project at Volvo Cars revising 
the security architecture of their cars.

Other students were taking just a two-year university diploma programme on 
distance.
Many had worked for many years, forgotten (or repressed) their upper-secondary 
maths.
Some of them panicked by just hearing the word \enquote{maths} even.
That makes a challenge trying to teach them basic security and giving them an 
idea for how number theory underpins modern cryptography.
Or even calculating the most basic performance metrics in an operating systems 
course.
Many of them expressed \enquote{I wish you would've been my maths teacher in 
school.}

I think that these students are the most interesting to teach.
The closest I get to that at KTH is in my programming courses, where the 
students have never programmed before\footnote{%
  Soon most will have programmed before coming to KTH, since now it's mandatory 
  in primary and secondary school.
}.
Some really struggle, even though they much higher grade averages and 
proficiency for maths.
But it's far from the same, the life and work experience of the distance 
students adds a lot of value.
At KTH I've heard a lot about the student who don't dare to turn on video, ask 
questions or are reluctant to speak during seminars.
That is much less a problem with more adult learners in the student population.
Those adult learners help remove that friction.

\subsection{Teaching experience}

% 2. Designs and develops courses appropriate for the programme and inspiring 
%    for the students.

I have developed a lot of teaching throughout my career.
I'll try to give some highlights.

I used to run a course on operating systems.
I designed it, developed all material.
One student wrote me an email about it, this is an excerpt:
\begin{quote}
  I really enjoyed this course and found it quite fascinating. There was a lot 
  of information to digest in that course book! :) My only complaint was that I 
  think the course book lacked worked examples, when it trying to explain 
  mathematical concepts.

  I particularly enjoyed the lab. I think that's a good way for a student to 
  learn more about a subject, by engaging them in a practical task. It makes it 
  more interesting and thus fun by doing some practical task, rather than just 
  accepting a whole bunch of concepts and acronyms at face value from the 
  course book.
\end{quote}
The lab was a lab assignment I developed.
I wrote a small memory management simulator, so that the students could 
experiment with different page-replacement algorithms.

Since I moved my lecturer position I have taught on the Computer Security 
course.
We have transformed the course to flipped classroom during the pandemic.
I used some videos that I had recorded for my Information Security course at 
Mid Sweden University a few years ago.
This year we noticed that very few students attended the lectures, so we 
surveyed the students to find out why and how they preferred their teaching.
Turned out that the cause was overload, the students had too much to do in 
other courses and could only cope by doing the mandatory things in the security 
course, namely the labs.
But among the survey responses I stumbled over this:
\begin{quote}
  I liked the pre-recorded videos when they were clearly explained with good 
  visualisations on the slides. For me, those are a fine alternative to 
  \enquote{in real life classes}. I feel like Daniel Bosk has delivered the 
    most enjoyable results in this regard.
\end{quote}

I have experimented with a lot of ways to improve the students' learning.
It ranges from the traditional lectures and labs to flipped classroom, 
hackathons and seminar driven modules.

The hackathon labs were full day labs where students had to solve a problem in 
a large group with the supervision of a tutor.
The idea was that the assignment was too difficult for individual students to 
solve on their own or in smaller groups.
We brought a projector to the lab, projected the screen with the code.
One student at a time wrote code (the driver in pair programming) while the 
rest of the class (the navigator) discussed and said what the driver were 
supposed to write.
It was an interesting concept, but required a skilled tutor present who could 
guide the group, ensure active participation from everyone and rotate the 
driver every now and then.

% 3. Demonstrates expertise in the subject matter (CS, security) and particular 
%    the ability to guide students into the subject.

Another construction that I find useful is the seminar driven modules that mix 
seminars with intermittent lab work.
I developed the idea in my Information Security course, for the authentication 
module.
Authentication is a challenging topic to teach, particularly because everyone's 
idea of authentication is username and password.
And those passwords should be at least eight characters, have upper and lower 
case, digits and special characters.
However, research has shown over and over that that's bad practice.
But since that's what the students are bombarded with in everyday life, I must 
must work really hard to break it.
The idea is that they read research on the problem, meet to discuss it and 
design some experiments to perform (e.g.~crack some passwords they thought were 
secure).
Then they perform the experiments before the next seminar where we discuss the 
results and possibly discuss new papers.

% 4. Draws on experience and pedagogical concepts to develop their teaching and 
%    supervision practice, with a focus on enhancing student learning.

Taken over all my teaching, the most influential theories of learning are 
constructivism and sociocultural theory.
We can see traces of them in both the hackathons and the seminar-driven 
teaching above.
The last two years, however, have been dominated by variation theory and 
phenomenography.
All of the theories complement each other.
But what I find very appealing of variation theory is that it pinpoints the 
necessary conditions of achieving a particular learning objective and allows me 
to analyse teaching (material) in detail in this regard.

\subsection{Collegial contributions}

% 5. Contributes to a collegial and collaborative educational culture, to 
%    programme development or thematic development, across the department, 
%    school of KTH.

% 6. Exchanges teaching experiences and ideas with colleagues and/or the wider 
%    higher education community.

I very much enjoy discussing pedagogy and didactics.
I have started a weekly pedagogy/didactic breakfast meeting together with Linda 
Kann in our division.
Otherwise, I try my best to contribute to the Cerise group, where I gave a talk 
on Marton's 
\citetitle{NecessaryConditionsOfLearning}~\citeyear{NecessaryConditionsOfLearning}.
I also contribute to KTH SoTL:
\begin{itemize}
  \item Bosk, Glassey: "When flying blind, bring a co-pilot", KTH SoTL'21.

  \item Bälter, Riese, Bosk, Glassey, Mosavat, Kann: "Question-based learning 
    with digital support in introductory Python programming courses", KTH 
    SoTL'21.
\end{itemize}
The co-pilot paper was later published at ITiCSE'21.

I also try to contribute to events like Storträffen and Öppen nätverksträff 
whenever time permits:
\begin{itemize}
  \item Storträffen (Autumn 2020), chaired a table on teaching online.
  \item Talk on Interaction in Zoom, Nationellt öppen nätverksträff (Spring 
    2021). Announced in pEECS and nationally. Participants from 11 
    universities.
\end{itemize}

I also interact on pedagogy with my former colleagues.
I convinced one of them to start using FeedbackFruits (I had worked on him on 
this since ScalableLearning existed).
Ironically enough, he managed to get LTI-integration at his institution (Mid 
Sweden University) before I got it at KTH (I still haven't gotten it).
We will do some joint work on how to best use FeedbackFruits in teaching to 
bootstrap our colleagues.

Finally, I participate in the Technology Enhanced Learning research group, 
primarily with Olle Bälter, Ric Glassey and Olga Viberg.
The current focus is on question-based learning and using the OLI platform.

