\mode*

\section{Introduction}

Let us start with an example from \textcite[Ch.~7, 
p.~239]{NecessaryConditionsOfLearning}.
\Textcite{Auction10yo} tried to teach 10-year-olds understand supply and demand 
through a fictive auction.
The teaching design was as follows:
The teacher varied supply (availability of the commodity, a type of toy 
machine) and demand (amount of money the students had at their disposal) one at 
a time.
Then varied both at the same time while changing the commodity from an ordinary 
toy machine to a limited edition toy machine.
This setup was supposed to be performed in five different classrooms.
The teaching was consistent with variation theory, except in one single 
respect:
the toy machine should've remained the same, not changed to the limited edition 
version.

Essentially no learning could be shown except for in one classroom.
In that classroom, the teacher had improvised (to counter the consequences of 
students' \enquote{irrational} actions) and as a result kept the commodity 
invariant (not changed the toy machine to the limited edition toy machine).
This subtle change made made it consistent with variation theory and the 
students actually learned the intended concept.

\Textcite{AuctionReplication} replicated this result by implementing the 
teaching design (the different auctions) as a computer program.
The only difference between the test and control group was a change of the 
image in the last auction (to the limited edition toy machine).
The results were the same.
\blockquote[{\cite[pp.~240--241]{NecessaryConditionsOfLearning}}][.]{%
  The difference between the two conditions was, of course, illusory. The 
  number of the identical items was the same as the number of different items 
  in each round, the base price was the same, and actually even the final 
  prices were very close to each other. The students in the comparison group 
  (the second one) could have neglected the visual differences between the 
  items, as they were completely irrelevant from the point of view of pricing. 
  But they turned out to have great difficulties in separating the unfocused 
  aspects (the kind of the commodity) from the focused aspects (demand and 
  supply), and the group dealing with the same commodity throughout (the first 
  one) clearly outperformed the group dealing with different commodities%
}

\paragraph{The idea of \acs*{TELS}}

Improvisation, as in this example, can go both ways.
In the example above, it fixed the problem; but it might just as well introduce 
a problem.
The example above was carried out as a learning study 
\parencite{LearningStudy}, meaning they had a pre- and post-test to measure 
students' learning.
This can easily be systematized using \ac{TEL}.

\paragraph{Why do we need this in normal education?}

If we don't do this in all our teaching, we can never really know what effect 
our teaching has had.
Consider the following that a student wrote in a reflection on their studies:
\blockquote[my translation][.]{%
  The most stressing right now is the seminar assignments in \textins{the 
  course} Algebra and Geometry.
  Every week we get new assignments and we have only a few days after the last 
  theory is covered in the lecture before hand in.
  To try to counter this, I spend time to watch videos before the lecture so 
  that I can understand the lecture immediately after, so that I can start 
  working on the seminar assignments right away instead of after having done a 
  lot of exercises in the book.
  The maths course is fast paced and there's a lot one must understand every 
  day.
  The most important is to continue to prepare for lectures and make sure I 
  don't get behind schedule%
}
The student in this case complements the teaching with video but removes 
exercises from the book due to the experienced tight time constraints.

Key message:
\begin{itemize}
  \item Programming teaching founded on learning theory and scientifically 
    verified to work

  \item CL students and school teachers participate in research to develop it

  \item Impact on schools' teaching in surrounding society
\end{itemize}

The problem:
\begin{description}
  \item[Why are you doing this?]
    %Because I can.
    We need scientifically founded, \emph{open}\footnote{CC-BY-SA} 
    material\footnote{In Swedish} that cover introductory programming.

  \item[What will it solve?]
    Well-researched teaching material.
    This material can be integrated into schools’ maths classes (mandated by 
    curriculum).
    School teachers and CL students are involved in continuous research 
    practice.

  \item[Why is it of importance?]
    It's open source science.
    Reinforces scientificness in teaching of programming.
    Improves teaching, reduces reinventing the wheel, improves sharing (open to 
    everyone).
\end{description}

Method, solution and results:
\begin{description}
  \item[What is the solution to the problem?]
    We take technology enhanced learning (scalable, asynchronous, reusable and 
    reproducible) to develop teaching.
    We combine this with learning studies~\parencite{LearningStudy} to improve 
    teaching content.

  \item[What is expected to be delivered in the project?]
    Scientifically founded teaching material introducing programming;
    research participation for CL students and school teachers.
\end{description}

Impact/applicability:
\begin{description}
  \item[What does it contribute to and for whom?]
    It contributes better teaching material in programming for everyone, in 
    Swedish.

  \item[Benefit to society?]
    Benefits quality of schools' teaching of programming in maths.
    Teachers get continuous exposure to research-in-practice for improving 
    teaching.
\end{description}

\begin{frame}
  \begin{block}<2,4>{Technology Enhanced Learning}
    \hitem Asynchronous + Interactive
    \quad
    \hitem Reproducible + Reusable
  \end{block}

  \begin{center}
    \Large\bfseries
    \onslide<1,2,4>{Technology Enhanced}
    \onslide<1-4>{Learning}
    \onslide<1,3,4>{Studies}
  \end{center}

  \begin{block}<3,4>{Learning Studies}
    \hitem Variation theory~\parencite{VariationTheory}
    \qquad
    \hitem Design-based research~\parencite{DesignBasedResearch}
  \end{block}

  \begin{onlyenv}<5>
    \begin{alertblock}{Output}
      \begin{itemize}
        \item Teaching material for intro programming,
        \item well-founded in pedagogic/didactic research.
      \end{itemize}
    \end{alertblock}
  \end{onlyenv}
\end{frame}

\section{Application and impact}

\begin{frame}
  \begin{block}<+>{Impact for KTH}
    \begin{itemize}
      \item Apply to teaching in intro programming courses.
        \begin{itemize}
          \item Many students (300+/year).
          \item Different \enquote{types} of students (three programmes).
        \end{itemize}
      \item Recruit TAs from CL programme.
        \begin{itemize}
          \item Well-versed in other theories of learning.
          \item They learn and practice this methodology.
          \item They get material to use (CC-BY-SA) in future.
        \end{itemize}
    \end{itemize}
  \end{block}

  \begin{block}<+>{Impact in society: open teaching for schools}
    \begin{itemize}
      \item Teaching is scalable by design (TEL/online).
      \item Must teach programming in maths.
      \item CL TAs move to schools when graduating.
      \item Schools can contribute data to and participate in research.
    \end{itemize}
  \end{block}
\end{frame}
