% What's the problem?
% Why is it a problem? Research gap left by other approaches?
% Why is it important? Why care?
% What's the approach? How to solve the problem?
% What's the findings? How was it evaluated, what are the results, limitations, 
% what remains to be done?

We outline the idea of Technology Enhanced Learning Studies.
This is a play on words combining \emph{Technology Enhanced Learning} and the 
\emph{Learning Studies} methodology~\cite{LearningStudy}.
The idea is to combine the them to provide reproducible systematization for 
researching and developing teaching.

Key message:
\begin{itemize}
  \item Programming teaching founded on learning theory and scientifically 
    verified to work

  \item CL students and school teachers participate in research to develop it

  \item Impact on schools' teaching in surrounding society
\end{itemize}

The problem:
\begin{description}
  \item[Why are you doing this?]
    %Because I can.
    We need scientifically founded, \emph{open}\footnote{CC-BY-SA} 
    material\footnote{In Swedish} that cover introductory programming.

  \item[What will it solve?]
    Well-researched teaching material.
    This material can be integrated into schools’ maths classes (mandated by 
    curriculum).
    School teachers and CL students are involved in continuous research 
    practice.

  \item[Why is it of importance?]
    It's open source science.
    Reinforces scientificness in teaching of programming.
    Improves teaching, reduces reinventing the wheel, improves sharing (open to 
    everyone).
\end{description}

Method, solution and results:
\begin{description}
  \item[What is the solution to the problem?]
    We take technology enhanced learning (scalable, asynchronous, reusable and 
    reproducible) to develop teaching.
    We combine this with learning studies~\cite{LearningStudy} to improve 
    teaching content.

  \item[What is expected to be delivered in the project?]
    Scientifically founded teaching material introducing programming;
    research participation for CL students and school teachers.
\end{description}

Impact/applicability:
\begin{description}
  \item[What does it contribute to and for whom?]
    It contributes better teaching material in programming for everyone, in 
    Swedish.

  \item[Benefit to society?]
    Benefits quality of schools' teaching of programming in maths.
    Teachers get continuous exposure to research-in-practice for improving 
    teaching.
\end{description}

