\section{Trying first or being told first?}

\Textcite{Szekely1950} concluded already in 1950 that letting the learners try 
to solve a problem before they were taught how to solve the problem led to 
significant improvement in both performance and retention compared to teaching 
the students first and letting them try afterwards.
This has later been reproduced by other 
studies~\parencites{NecessaryConditionsOfLearning}[see for 
instance][]{BransfordSchwartz1999}.

\Textcite{NecessaryConditionsOfLearning} also mentions the study by 
\citeauthor{Szekely1950}, but studies the problem further.
He analyses material from the TIMSS 1999 Video Study\footnote{%
  URL: \url{https://www.timssvideo.com/}.
}~\parencite[see \eg][]{ClosingTeachingGap}, particularly he compares the US 
and Japanese classrooms.
The problem at hand is that the American students perform worse than the 
Japanese, who are among the best performers in the world.
In the US, in the typical lesson, the teacher first introduced a particular 
method for solving a certain class of problems.
After the demonstration, the teacher gave the students problems of that same 
class of problem to practice solving using the method that the teacher 
introduced.

In a typical Japanese lesson, on the other hand, the teacher first introduced a 
rather complex problem and invited the students to try to solve 
it---individually or in small groups.
After a while the teacher asked the students to present their attempted 
solutions, which could be compared to the others' attempts.
They would conclude that one way of solving the problem is more powerful than 
the others, then they would practice this method on other problems in the same 
class of problems.

There are two things we'd like to highlight here.
First, we can probably see some of the same effect as \textcite{Szekely1950} 
detected.
Second, \textcite{NecessaryConditionsOfLearning} points out that the patterns 
of variation and indifference (as in variation theory, \cite[see 
\eg][Ch.~2]{NecessaryConditionsOfLearning}) differ.
In the American classroom, the method is invariant whereas the problems vary.
In the Japanese classroom, at first, the problem is invariant and methods vary, 
followed by method being invariant while problems vary.

According to the variation theory of
learning~\parencite[see \eg][Ch 3]{NecessaryConditionsOfLearning},
for a student to be able to learn they must experience variation in the 
critical aspects of what is being taught while other aspects remain invariant.
In the Japanese classrooms, the method varies first, while the problem is :
invariant.
This means that the focus is on the necessary aspects of the method.
In the American classroom, however, this focus is never covered, they simply 
focus on the necessary aspects of the problems for a given method.
Now, \textcite{NecessaryConditionsOfLearning} says \blockquote[p.~183][.]{%w
  the learners pay attention to that which varies. If this is true, the 
  Japanese students will focus primarily on ways of solving problems, while the
  American students will primarily focus on how problems among certain kinds of
  problems differ from each other, mostly in terms of numerical values%
}
He continues: \blockquote[{\cite[p.~183, original 
emphasis]{NecessaryConditionsOfLearning}}][.]{%
  The other difference is that while the American students are \emph{told} how 
  to solve a certain kind of problem, the Japanese students have to try to 
  \emph{find} a way of solving a particular problem themselves. Even this 
  difference can be formulated in terms of differences in the pattern of 
  variation and invariance. When you are told how to solve a problem, the one 
  who tells you makes all the distinctions that have to be made. In the best 
  case you learn to recognize what has been discerned. You might learn how to 
  solve that particular kind of problem, but you cannot possibly see the 
  solution as \emph{a} way of solving that kind of problem.
  In fact, there is no way of separating that kind of problem and the solution 
  presented by the teacher unless you come across another way of solving the 
  same kind of problem. When trying to solve a problem on your own---if you 
  really try to solve it---there is probably a change (difference) in how the 
  problem appears to you.
  You develop a way of seeing it. This is true whether or not you actually 
  manage to solve it. If you do not and you eventually see how it is solved by 
  someone else (the teacher or a classmate), there will be a contrast between 
  the canonical way of solving it and your own. Your own, perhaps less elegant 
  or even failed attempt, will enable you to see the solution much more 
  clearly. It will have a particular meaning for you%
}

Are there more arguments for trying first or for being told first?
\Textcite[p.~214]{NecessaryConditionsOfLearning} concludes that both are 
needed:
\blockquote{%
  The present line of reasoning is contrary to the received wisdom that first 
  you learn the parts and thereafter you put them together to make the whole. 
  It is also contrary to the received wisdom that the general concept or 
  principle is the whole and that the instances are the parts. According to the 
  view advocated here, each instance, each case is a whole from which aspects, 
  features of a general nature or various parts with features of general 
  nature, are discerned%
}.

\Textcite{BransfordSchwartz1999} also
\blockquote[{\cite[p.~217]{NecessaryConditionsOfLearning}}]{%
  demonstrate\textins{d} that after having tried to solve a problem, students 
  learn much more easily from being told how to solve it than they would in the 
  case where the same total amount of time is used for telling them how to 
  solve it%
}.
Now, some things cannot be found out, for instance conventions or terminology.
Those must be told.
However, these are not made up out of thin air, there are reasons behind them 
being chosen.
But while the students can develop an understanding for how things are named, 
it's hard for them to find out by themselves what was the actual outcome of a 
standardization process, they must be told.
In conclusion, we can conclude from the \textcite{BransfordSchwartz1999} study 
that \blockquote[{\cite[p.~218]{NecessaryConditionsOfLearning}}]{%
  \enquote{\textins{t}rying to find out} is a better preparation for being told 
  than \enquote{being told.} The combination of two ways of learning is more 
  powerful than either of them%
}.

However, there are more studies showing that it's good for the students' 
learning to try first, even if it results in a failure.
\Textcite{kapur2008productive,kapur2010productive,kapur2012productive} has a 
series on \enquote{productive failure}.
Two groups of students, one under a \enquote{being told} condition, the other 
under a \enquote{finding out} condition.
The total time spent were the same for both conditions.
In the \enquote{being told} condition the students were first told the method 
to solve the problems and then they practiced it.
In the \enquote{finding out} condition they first tried to solve a problem, 
they spent half the time trying to solve it by themselves.
The other half was the same as the \enquote{being told} condition, except 
shorter time for practicing.
The problems in this case was on the topic of analysing distributions in data.
\blockquote[{\cite[p.~219]{NecessaryConditionsOfLearning}}]{%
  Kapur distinguished between different outcomes. Both groups
  did equally well on items measuring \enquote{procedural fluency} (i.e., in 
  computing the variance). On items measuring data analysis, conceptual insight 
  and transfer, the \enquote{finding out} group did better than the 
  \enquote{being told} group
}.

So, while these studies show that trying to find out (followed by being told) 
results in better learning than either of these experiences alone, or both in 
the reverse order; why is that?
According to \textcite{NecessaryConditionsOfLearning} it's because the 
\enquote{finding out} phase allows the students to introduce sufficient 
variation to themselves, so that when they're finally being told they have 
discerned the necessary aspects.

We should note that there is an alternative way of phrasing the explanation.
When being told the method first, the focus is on memorizing (surface 
learning).
When trying first, being told later, the focus is on the necessary aspects 
(deep learning\label{DeepLearning}).
\Textcite[Ch.~5]{NecessaryConditionsOfLearning} shows that indeed the 
difference between surface and deep learning is the use of different patterns 
of variation and invariance.

