\section{Testing}

Considering the results above, it is good for the students' learning to let 
them try solving a problem on their own before being told.
Another reason for letting the students attempt a problem before teaching them 
how to solve it is that we can then evaluate our teaching.
To properly evaluate any teaching we must measure any difference in knowledge 
that results from teaching, through a pre- and 
post-test \parencite{NecessaryConditionsOfLearning}.
The idea discussed here is how to align the pretest with the \enquote{finding 
out} part of teaching suggested above.

Now, if we consider learning to be getting better at discerning the necessary 
aspects; which is what deep learning is about according to 
\textcite{NecessaryConditionsOfLearning}, as mentioned in \cref{DeepLearning}.
How do we measure learning?
Well, certainly, we shouldn't discern those necessary aspects for the students, 
we want them to do it, that's the ability we want to assess.
Let us consider an example from assessing students' learning of Newtonian 
mechanics.
This assignment is from the nationwide physics test for year 11 in Sweden 
\parencite{NecessaryConditionsOfLearning}:
\blockquote{%
  In an experiment with a ball, it is found that when the ball falls, it is 
  affected by the air with a braking force F, which is proportional to the 
  velocity of the ball, v, that is, F = kv where k in this case is 0.32 Ns/m. 
  What would the final velocity of the ball be if it were dropped from a high 
  altitude? The ball’s mass is 0.20 kg.%
}
The problem with using this question for assessment is that 
\blockquote[{\cite[p.~90]{NecessaryConditionsOfLearning}}]{%
  \textins{t}he learners do not have to discern what aspects of the event have 
  to be taken into consideration: they all are pointed out in the question.
  But they have to know that the gravitational force acting on the falling body 
  is mg and that if you drop something from a high enough altitude, the 
  gravitational force and the breaking force will eventually balance each other 
  and the body will continue falling at constant speed.
}
A better question for assessing what the students discern in terms of Newton's 
first law would be the following: 
\blockquote[{\cite[p.~237]{johansson1985approach}}, through 
{\cite[p.~91]{NecessaryConditionsOfLearning}}]{%
  A car is driven at a high constant speed on a motorway. What forces act on 
  the car?%
}
In this question, we simply discern for the students that there are several 
forces acting on the car.

For our discussion, it is relevant to point out that during the pretest the 
students should be able to learn as little as possible\footnote{%
  Preventing them from learning anything in this situation is 
  information-theoretically impossible.
}.
Clearly they learn more from the first question above than they do from the 
second.
If they are provided with a solution, \eg for self-grading, they learn even 
more.

To measure the learning effect from any feedback, we can do the following.
We divide the participants into six groups:
Two that will take a diagnostic test with multiple choice questions, but 
without any feedback (D1 and D2).
(This reduces the possibility for learning to the question itself.)
The next two groups (F1 and F2) take the same test, but they also receive 
feedback that they can learn from.
The last two groups (O1 and O2) will also take the same test, but as open 
questions.
The questions on the test are divided such that D1, F1 and O1 receive the same 
questions in the same order (over the pre- and post-test); D2, F2 and O2 
receive the same questions but in the reverse order (over the pre- and 
post-test).

To assess learning properly, we must use open 
questions~\parencite{NecessaryConditionsOfLearning}.
But considering that our teaching approach is question based, we would like to 
know to what extent we can use our \enquote{teaching questions} also for 
assessment.
With our construction of pre- and post-tests above, we can measure how much is 
learned from the pre-testing.

We must measure learning taking place over different periods.
We must have a pre- and post-test for the whole course as well as individual 
modules.

