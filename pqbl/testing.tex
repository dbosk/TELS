\section{Testing}

When teaching, it is important that the teacher let the learners try for 
themselves.
\Textcite{Szekely1950} concluded already in 1950 that letting the learners try 
to solve a problem before they were taught how to solve the problem led to 
significant improvement in both performance and retention compared to teaching 
the students first and letting them try afterwards.
Another reason for letting the students attempt a problem before teaching them 
how to solve it is that we can then evaluate our teaching.
To properly evaluate any teaching we must measure any difference in knowledge 
that results from teaching, through a pre- and 
post-test~\parencite{NecessaryConditionsOfLearning}.

According to \textcite{NecessaryConditionsOfLearning} and the variation theory 
of learning, for a student to be able to learn they must experience variation 
in the critical aspects of what is being taught.
For example seeing which type of problem a method works for and not.
(It is important to highlight examples of both \enquote{works} and 
\enquote{doesn't work}.)

We would like to extend the question-based learning approach in the OLI 
initiative into a purely question-based learning approach.
This aligns nicely with the theory and we hypothesise that it will do in 
practice too.
Why?
Each question and its answers must introduce variation in a critical aspect 
(according to \cite{NecessaryConditionsOfLearning}).
The student must first approach the problem and will be told only afterwards 
(\cite{Szekely1950} and \cite{NecessaryConditionsOfLearning}).
Learning is made possible when the \emph{motivations} for the different choices 
for an answer are presented (suitable variation is introduced).

Finally, to measure learning, we must do pre- and post-tests.
When measuring learning, we need to first measure what they know before we've 
taught them anything.
However, according to our hypothesis, the students might learn something from 
the initial testing (particularly if provided with feedback), so we must try to 
control for this (and measure any effect).

We will divide the participants into six groups:
Two that will take a diagnostic test multiple choice questions but without any 
feedback (D1 and D2).
(This reduces the possibility for learning.)
The next two groups (F1 and F2) take the same test, but they also receive 
feedback that they can learn from.
The last two groups (O1 and O2) will also take the same test, but as open 
questions.
The questions on the test are divided such that D1, F1 and O1 receive the same 
questions in the same order (over the pre- and post-test); D2, F2 and O2 
receive the same questions but in the reverse order (over the pre- and 
post-test).

To assess learning properly, we must use open 
questions~\parencite{NecessaryConditionsOfLearning}.
But considering that our teaching approach is question based, we would like to 
know to what extent we can use our \enquote{teaching questions} also for 
assessment.
With our construction of pre- and post-tests above, we can measure how much is 
learned from the pre-testing.

We must measure learning taking place over different periods.
We must have a pre- and post-test for the whole course as well as individual 
modules.

