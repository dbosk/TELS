% Daniel, do you think you could add a some information about active and 
% passive learning in the subsection about the doer effect in the B_background 
% part of the manuscript?

\section{Active and passive learning activities}

The Encyclopedia of the Sciences of Learning defines \emph{active learning} as 
\blockquote[{\cite[p.~77]{EncyclopediaActiveLearning}}][.]{%
  instructional techniques that allow learners to participate in learning and 
  teaching activities, to take the responsibility for their own learning, and 
  to establish connections between ideas by analyzing, synthesizing, and 
  evaluating.
  \textelp{}
  The learner’s role is being involved in learning activities such as 
  discussions, reviewing, and evaluating, concept mapping, role playing, 
  hands-on projects, and cooperative group studies to develop higher-order 
  thinking skills such as analysis, synthesis, and evaluation%
}
By contrast, a student's role in \emph{passive learning} is simply to watch, 
listen or read.

There is a body of research aimed at determining whether active learning is 
better than passive learning.
The consensus seems to be leaning towards that active is better than passive, 
but this is difficult to say in general terms.
For instance, there are studies~\parencite{ControlledTrialActiveVsPassive} that 
observed no difference beyond reducing passive delivery of content by 50\% (the 
other 50\% being the active part; see also \cite{BransfordSchwartz1999}, for 
more similar results), there are other studies~\parencite[see Table 
1 in][]{ActiveVsPassiveTeachingStyles} that only observed improvement in 
retention for active learning.
There are other studies~\parencite[see Table 1 in][]{ActiveVsPassiveTeachingStyles} showing that student like active learning 
better, while other studies~\parencite{ActualLearningVsFeeling} show that 
students perceive passive approaches as better for learning despite the active 
actually being better in that particular case.

There are even studies~\parencite{SmithSmith+2015+86+99} making valid arguments 
for passive learning.
However, the active methods used in the situation of 
\textcite{SmithSmith+2015+86+99} are social: discussion forums, synchronous 
teaching sessions.
But as we will see, students can be active yet not interact with others.
Question-based learning might be a better form of active learning for the 
passive students of \citeauthor{SmithSmith+2015+86+99}, so that they can be 
active without social interaction.

We would like to highlight two tracks of research:
First, one initiated by \textcite{Szekely1950}.
Second, a series of papers started by \textcite{koedinger2015}.
Both treating active in its simplest form: just actively working with problems 
in the topic at hand.

\Textcite{Szekely1950} let the learners try to solve a problem before they were 
taught how to solve the problem and found that this led to significant 
improvement in both performance and retention compared to teaching the students 
first and letting them try afterwards.
This has later been reproduced by other 
studies~\parencites{NecessaryConditionsOfLearning}[see for 
instance][]{BransfordSchwartz1999}.

\Textcite{koedinger2015} studied the effect of certain active elements (solving 
problems with automated feedback, not social) compared to the passive elements 
(video lectures) of a large MOOC course.
They concluded that \enquote{extra doing} (one standard deviation increase) was 
more than six times that of \enquote{extra watching or reading}.
In a series of follow-up 
studies~\parencite{koedinger2016doer,koedinger2018doer,van2021doer} they 
concluded that this is a causal relationship.
